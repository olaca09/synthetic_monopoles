\documentclass{article}
\usepackage[utf8]{inputenc}
\usepackage[swedish]{babel}
\usepackage{amsmath}
\usepackage{graphicx}
\usepackage{hyperref}
\usepackage{appendix}
\usepackage{listings}
\usepackage[super, square]{natbib}
\usepackage[section]{placeins}
\usepackage{pdfpages}
\lstset{
language=Python,
breaklines}

\title{Code Outline}
\author{Ola Carlsson}

\begin{document}

\maketitle
This is an outline for the code to be written. Ignore the saving of states for now, that's
a later problem. Possibly keep the saving of the external field.
\begin{description}
\item[Some main script] A script for execution, but some of the scripts below should be
        callable separately.
\item[Magnetic field generation] This should save magnetic fields as a separate file to be
        read, numpy has native savefiles.
\item[Eigenvalue approximation] This is calling some eigenvalue-finding function on the
        matrix in question. For unique identification the energies are assumed to never
        cross.
        %Store in dictionary to reduce unnecessary calculation.
\item[Eigenstate calculation] Returns eigenfunctions to the fast Hamiltonian, the same
        function as above?
\item[Gauge field calculation] This acts the differentiated matrix on the eigenstates and
        thus calculates the synthetic
        scalar potential, save to file. 
\item[Acceleration function] A function to return the acceleration for some set of
        parameters, i.e. the function used in the ODE solver. This is big. Takes external
        field, coordinate velocities, coordinates. Calls calculation of eigenvalues and
        eigenstates in neighbouring sites, gauge field calculation.
\item[ODE solver] This should integrate the acceleration to generate the path traversed.
\item[Visualization] This will be some visualisation of the result, as fancy as possible.
        If time permits animation is a possibility. If parameter space is done the
        monopoles could perhaps be shown as well.
\end{description}
Data treatment can be done in xarray so that labels can be used in arrays. This turns out
to be ineffective, switch back to full numpy ithink

TODO: index field after position not point, and modify the rest of the code accordingly
\end{document}

