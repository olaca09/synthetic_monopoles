\documentclass [a4paper]{article}
\usepackage{epsfig}
\usepackage{epsf,graphics,graphicx}
%\renewcommand{\baselinestretch}{1.0}
\begin{document}
\section*{{\underline{Project}}: Classical motion in artificial magnetic monopole fields}
\vskip 0.3 cm
\subsection*{Background} 
Berry demonstrated that a general quantum system undergoing adiabatic evolution 
may pick up a nontrivial phase of purely geometric origin \cite{berry84}. For a spin in a slowly 
rotating magnetic field this phase turns out to be proportional to the flux of an artificial magnetic 
monopole sitting at the energy crossing at the origin (corresponding to vanishing magnetic 
field). Due to the elusiveness of magnetic charges in standard electromagnetism, this result 
has triggered numerous studies in adiabatic evolution of spin systems \cite{shapere89}.  

Real-space realizations of artificial magnetic monopoles have been demonstrated in spin ice 
\cite{castelnovo08} and Bose-Einstein condensates \cite{ray15}. Similarly, Berry artificial 
monopoles in parameter space can be mapped nontrivially to physical space via a nonuniform 
magnetic field that drives the adiabatic spin system. This opens up for  the experimental 
study of classical trajectories in the presence of magnetic monopoles.  

\subsection*{Scope and purpose} 
The aim of the project is to derive classical equations of motion for the centre of mass 
(CM) of a particle with spin moving in an arbitrary inhomogeneous magnetic field. A key ingredient is to 
examine the influence of internal spin-spin interaction on the trajectories of the CM. The 
spin-spin interactions are known to modify the distribution of artificial monopoles in the
parameter space, for example 
splitting magnetic charges yielding monopoles located at points where the magnetic field is 
nonzero \cite{eriksson20}.  

\vskip 0.3 cm 
\noindent
The key objectives of the project are: 
\begin{itemize}
\item to derive and (numerically) simulate classical CM equations of motion of a composite 
system (such as an atom or molecule) consisting of pairs of interacting spins; 
\item to estimate parameter ranges needed to see effects of the internal spin-spin 
interaction on the CM trajectories.
\item to discuss the plausibility of realising a system within those parameter ranges.
\end{itemize}
The form of spin-spin interaction is restricted to an Ising model interaction.
Spin-$\frac{1}{2}$ constituents only are
considered. 

The overall purpose is to develop a physical setting that allows for direct experimental studies 
of artificial magnetic monopoles in real space. 
\vskip 0.5 cm
\subsection*{Method}
Through the Born-Oppenheimer approximation an effective Hamiltonian for the adiabatic
evolution of a spin-system in an external magnetic field can be found. This is dependent on a
scalar potential akin to an electric potential, as well as a vector potential analogous to a
magnetic field\cite{berrylim90}.%Uppdatera till ursprungskällan?
This vector potential can be seen as an artificial magnetic field giving rise to the
geometric Berry phase, and it is the dynamics arising from this field as well as the scalar
potential that are to be simulated. Evaluating these potentials for a Hamiltonian
containing the interactions of the system, the effective Hamiltonian is used as a
basis for classical Hamiltonian dynamics. 

Numerically simulating the resultant differential equations of motion for select parameters
is carried out by development of a suitable Python-script. Auxiliary tools such
as graphical illustration and evaluation of the external magnetic field are also necessary.

With these tools in place the effect of different parameter choices on the dynamical
evolution can be explored and analysed. Of certain importance is the choice of external magnetic
field, as some properties (e.g. points of zero magnitude) are suspect to yield interesting
results but must be found within the context of realisable fields (e.g. zero divergence).

\subsection*{Thesis outline}
As a rough outline the following sections will be included in the report:
\subsubsection*{Abstract}
This will be a regular abstract.
\subsubsection*{Sammanfattning}
This will be a Swedish version of the abstract.
\subsubsection*{Introduction}
This will be a short summary of the background and a thorough statement of the problem and
aim of the thesis. In particular some example realisation of the simulated system can be
sketched, and some context of magnetic monopoles. Some summary over the principle of geometric
phase could also fit in here.
\subsubsection*{Background}
This will be mostly a theoretical background considering the work done previously on
geometric phase and its peculiarities (for example the dependence of artificial monopole positions on
field and system characteristics).
\subsubsection*{Theory}
A section taking a more in depth look at the theory necessary for this specific problem, culminating in the
equations of motion. One area of interest is specifying the approximations
necessary.
\subsubsection*{Simulation}
Here details of the implementation of the numerical simulation can be described. Complete
code will follow in an appendix.
\subsubsection*{Results}
As results mainly the exploration of the parameter dependence of the resultant dynamics
should here be considered. Comparisons of different systems would be of interest as well as
less comparative estimations of say the orders of magnitude of parameters necessary for
appreciable effects. All of this is preferably done through presenting select simulations
and their parameters.
\subsubsection*{Discussion}
Here a summary and discussion of the relevance of the above results can be carried out. In
particular the plausibility to realise systems with the required parameters, one of the key
objectives, should be considered. This is also the somewhat appropriate space for
reflecting on encountered hindrances and their solutions or negative effects on the final
result, for example related to the numerical simulation.
\subsubsection*{Outlook}
This will be a discussion of unresolved issues and areas of improvement for this project,
but perhaps also possible further projects, experimental or theoretical, related to or suggested by the results achieved.
\subsubsection*{Conclusions}
This will be a short summary of the results and discussion above, so as to conveniently
summarise the thesis when combined with the abstract and introduction.
\subsection*{Time plan} 
The project is carried out during spring 2022. A draft of the time plan with notable
milestones is as follows: 

\begin{itemize} 

\item[-] Literature study, 10 Jan. - 23 Jan.

\item[-] Formal start of course 25 Jan.

\item[-] Deriving equations of motion, construction of model (external magnetic field 
configuration), 24 Jan. - 13 Feb.

\item[-] Development of numerical code for classical trajectories, 14 Feb. - 13
        Mar.

\item[-] Deadline for complete project plan 25 Feb.

\item[-] Numerical simulations and interpretation, 14 Mar. - 10 Apr.

\item[-] ''Mittavstämning'', a five minute presentation on project status, 8 Apr.

\item[-] Writing report, 28 Mar. - 5 Jun. First draft has deadline 13 May, a first version
        has deadline 20 May and the final report
        has deadline 5 June.

\item[-] Feedback on report 27 May.

\item[-] Preparation and presentation, 16 May - 5 Jun.

\item[-] Symposium with poster and presentation, 30 and 31 May

\end{itemize}

\subsection*{Supervision}
Weekly consultations with the supervisor, to a large extent by means of zoom meetings, will take place
predominantly Thursdays. The aim is to provide if applicable the week's results, say drafts
of documents or problems encountered, some time in advance so that the meetings remain
effective and their subjects clear.

\begin{thebibliography}{99}
\bibitem{berry84} M. V. Berry, 
Quantal Phase Factors Accompanying Adiabatic Changes, 
Proc. R. Soc. London Ser. A {\bf 392}, 45 (1984).  
\bibitem{shapere89} A. Shapere and F. Wilczek, 
{\it Geometric phases in physics} (1989). 
\bibitem{castelnovo08} C. Castelnovo, R. Moessner, and S. L. Sondhi, 
Magnetic monopoles in spin ice, Nature (London) {\bf 451}, 42 (2008).
\bibitem{ray15} M. W. Ray, E. Ruokokoski, K. Tiurev, M. M\"ott\"onen, and D. S. Hall, 
Observation of isolated monopoles in a quantum field, 
Science {\bf 348}, 544 (2015).
\bibitem{eriksson20} A. Eriksson and E. Sj\"oqvist, 
Monopole field textures in interacting spin systems, 
Phys. Rev. A {\bf 101}, 050101(R) (2020). 
\bibitem{berrylim90} M. V. Berry, R. Lim,
The Born-Oppenheimer electric gauge force is repulsive near degeneracies,
J. Phys. A: Math. Gen. {\bf 23}, L655 (1990).
\end{thebibliography}
\end{document}





