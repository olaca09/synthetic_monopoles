\documentclass[a4paper]{article}

\usepackage[utf8]{inputenc}
\usepackage[T1]{fontenc}
\usepackage{textcomp}
\usepackage[swedish]{babel}
\usepackage{amsmath, amssymb}
\usepackage[arrowdel]{physics}
\usepackage{pdfpages}
\usepackage{graphicx}
\usepackage{hyperref}
\usepackage{appendix}
\usepackage[super, square]{natbib}
\usepackage{listings}
\usepackage[section]{placeins}
\usepackage[bb=dsserif]{mathalpha}
\usepackage[margin=1.2in]{geometry}

\usepackage{tikz}

\newcommand{\R}{\ensuremath{\mathbb{R}}}
\newcommand{\C}{\ensuremath{\mathbb{C}}}
\newcommand{\Arg}{\ensuremath{\text{Arg}}}
\newcommand{\U}{\ensuremath{\mathbb{\cal U}}}

\usetikzlibrary{arrows.meta}
\usetikzlibrary{decorations.pathreplacing}
\usetikzlibrary{decorations.markings}
\usetikzlibrary{patterns}
\lstset{language=Python, breaklines, literate=%
{å}{{\r{a}}}1
{ä}{{"a}}1
{ö}{{"o}}1
{Å}{{\r{A}}}1
{Ä}{{"A}}1
{Ö}{{"O}}1
}

\title{Derivations for synthetic monopoles}
\author{Ola Carlsson}


\pdfsuppresswarningpagegroup=1

\begin{document}
\section{System description}
\subsection{Coordinates and quantities}
Consider a dumbbell-like system consisting of two equal masses at a distance \(l\) from one
another, and let \(m\) be the total mass. The system can be freely translated and rotated
throughout space, so let \(x\), \(y\), \(z\)  be the position
of the centre off mass, and \(\theta_r, \varphi_r\) be the polar and azimuthal angle
respectively of the axis connecting the two masses. Notate these coord2inates compactly as the
vector \[
\va{r}
= \begin{pmatrix} x\\ y \\ z\\ \vartheta_r \\\varphi_r\end{pmatrix}
.\] 
Fix the angles such that a polar angle of \(\varphi_r = m_0\) implies a dumbbell parallell
to the \(z\)-axis and so that an azimuthal angle of \(\vartheta_r = 0\) implies that the dumbbell
axis lies in the \(xz\)-plane. 

Now, consider also each of the masses of the dumbbell to carry spin, intrinsic angular
momentum, of size \(\frac{1}{2}\) each. %more precise formulation?
 The state of the spin components must be
described quantum mechanically, so let \(\ket{s, m'}\) denote the state of the system with
\textit{total} spin magnitude squared \(s(s+1)\hbar^2\) and \textit{total} spin measured
along the axis of the dumbbell \(\hbar m'\). An external field \(\va{B}\) is present, which
we can describe by its magnitude \(B\) and its angular direction \(\vartheta_B, \varphi_B\)
in analogue with the angles defined above.

\subsection{The Hamiltonian}

The time evolution of such a system is governed in both classical and quantum mechanics by
its Hamiltonian. Since spin is the epitome of a phenomena demanding a quantum mechanical
interpretation we have no choice but to model the whole system quantum mechanically.%, at
%least for now?
The Hamiltonian which will be assumed for the system here is:

\begin{equation}\label{eq:Hamil0}
        \mathbb{\cal H} = \sum_{i=1}^{5} \frac{\va{p_i}^2}{2m_i} + 4JS^{(1)}_{z}S^{(2)}_{z} -
        \gamma\va{B}\cdot \va{S}.
\end{equation}

The first sum is over the five degrees of translational and rotational coordinates in \(\va{r}\)
. The momentum and angular
momentum operators are
taken to be \(p_i = i\hbar \partial_i\label{def:mom}\) with \(\partial_i\) as the derivative with respect
to the
corresponding coordinate. Note that it is not a priori clear that the effective masses
\(m_i\) for
all degrees of freedom are the same, but we can until later note that at least the first
three are equal to \(m\).

For the potential energy the spin-spin interaction is taken to be of Ising form, which is
the first term after the sum, while the interaction between spin and magnetic field is
considered in the final term. %correct name of the final term?
The parameters \(J\) and \(\gamma\) are the strengths of both
of these interactions, while the operators \(\va{S}\) and \(S^{(n)}_{z}\) are respectively
the one
related to the total spin of the system and the spin in the z-direction of one of the
system
components.

\subsection{Effective mass}
To clearly see the values of the effective masses paired with the rotational momenta a quick derivation
of the kinetic part of the Hamiltonian is in order. The kinetic energy related to rotation
is of the form \[
K_{rot}= \frac{m}{4}\pqty{\frac{l}{2}}^2\pqty{\dot \vartheta_r^2 + \dot \varphi_r^2}
,\] 
which is the same as the relevant terms of the Lagrangian. The quantum mechanical momenta
correspond to the momenta received from differentiating the classical Lagrangian, and as
of %fråga Erik
such we have in the classical picture that 
\begin{align*}
        p_4 &= \frac{\partial K}{\partial \dot \vartheta_r} = \frac{ml^2}{8}\dot \vartheta_r\\
        p_5 &= \frac{\partial K}{\partial \dot \varphi_r} = \frac{ml^2}{8}\dot \varphi_r\\
.\end{align*}
Performing the Legendre transform from the Lagrangian to the Hamiltonian yields:
\begin{align*}
        \mathbb{\cal H}_{rot}= p_4 \vartheta_r + p_5 \varphi_r - K = \frac{\dot
        p_4^2 + \dot p_5^2}{2} \frac{8}{ml^2}
.\end{align*}
It is then clear that the effective masses to be used in equation \ref{eq:Hamil0} are:
\begin{align*}
        m_i = \begin{cases}
                m & i = 1, 2, 3\\
                \frac{ml^2}{8} & i = 4, 5
        \end{cases}
.\end{align*}

\subsection{Rotation matrices}
The potential energy operators will be of great use in some matrix form, so let the spin
state be described as a coordinate vector in the basis \((\ket{0, 0}, \ket{1, -1}, \ket{1, 0},
\ket{1, 1})\). In the special case where the axis of the dumbbell (henceforth ''Ising
axis'') is parallell to the direction of the magnetic field \(\va{B}\), it is then clear that the operators take the form:
\begin{align}
        \va{B}\cdot \va{S} &= B\hbar \label{eq:BSmat}
        \begin{pmatrix}
        \dmat{0,-1,0,1}
        \end{pmatrix} \\
        S^{(1)}_{z}S^{(2)}_{z} &= \frac{\hbar^2}{4} \label{eq:SSmat}
        \begin{pmatrix} 
        \diagonalmatrix{-1,1,-1,1}\end{pmatrix} 
.\end{align}

The second matrix follows from the well known representation of a two-component
spin-\(\frac{1}{2}\) system as singlet and triplet states: Let \(\ket{m_1, m_2}_1\) be the
state with spin-\(z\) number \(m_1\) for the first component and \(m_2\) for the second
component. Then
\begin{align*}
    \ket{0, 0} &= \frac{1}{\sqrt{2}}\bqty{\ket{\frac{1}{2}, -\frac{1}{2}}_1 -
                \ket{-\frac{1}{2}, \frac{1}{2}}_1}\\
    \ket{1, 1} &= \ket{\frac{1}{2}, \frac{1}{2}}_1\\
    \ket{1, 0} &= \frac{1}{\sqrt{2} }\bqty{\ket{\frac{1}{2}, -\frac{1}{2}}_1 +
            \ket{-\frac{1}{2}, \frac{1}{2}}_1}\\
    \ket{1, 1} &= \ket{\frac{1}{2}, \frac{1}{2}}_1
.\end{align*}

For the matrix \ref{eq:BSmat} above we assumed that the Ising axis was parallell to the magnetic field.
This is not necessarily the case, so we must find a suitable rotation matrix. This should
express a spin
state, in the previously given basis, in the basis with eigenstates to measurement along the
magnetic field. The basis vectors are to be rotated through the polar and azimuthal angles
\[
\vartheta = \vartheta_B - \vartheta_r \text{, }\quad \varphi = \varphi_B - \varphi_r
.\] 
Consider therefore first a rotation of the \textit{state} vectors, which can then easily
be inverted to receive the forwards transformation also necessary for the transformation
of operator matrices. The inversion process is but a complex
conjugation since the operator in question is unitary. It can be shown %cite Sakurai
that the rotation about three Euler angles \(\alpha\), \(\beta\), \(\delta\) of a state is
given  by the matrix with elements as:
\begin{align*}
        \U_{m'm''} = \bra{s,
        m'}e^{\frac{-iS_z\alpha}{\hbar}}e^{\frac{-iS_y\beta}{\hbar}}e^{\frac{-iS_z\delta}{\hbar{}}}\ket{s,
m''}
.\end{align*}
Here \(s\), \(m'\), \(m''\) are spin quantum numbers of the system, which in the more general
case can be replaced by angular momentum quantum numbers. The rotations \(\alpha\),
\(\beta\) and \(\delta\) are done about the \(z-\), \(y-\) and then \(z-\) body axes of
the system in turn. Since the system considered here is rotationally symmetric about the
Ising axis the final rotation \(\delta\) is superfluous and thus will be discarded.
Identifying the angles \(\alpha = \varphi\) and \(\beta = \vartheta\) it can further be
shown that the exponential operators amount to:
\begin{align*}
        \U = \pmqty{
        1 & 0 & 0 & 0\\
        0 & \frac{e^{-i\varphi}}{2}(1+\cos(\vartheta)) &
        \frac{e^{-i\varphi}}{\sqrt{2}}\sin(\vartheta) &
        \frac{e^{-i\varphi}}{2}(1-\cos(\vartheta))\\
        0 & -\frac{1}{\sqrt{2}}\sin(\vartheta) & \cos(\vartheta) & \frac{1}{\sqrt{2} }
        \sin(\vartheta)\\
        0 & \frac{e^{i\varphi}}{2}(1-\cos(\vartheta)) &
        -\frac{e^{i\varphi}}{\sqrt{2}}\sin(\vartheta) & \frac{e^{i\varphi}}{2}(1 +
        \cos(\vartheta))
}
.\end{align*}
An operator matrix \(A\) transforms under rotation as \(A_{rot} = \U A\U^{\dagger}\)
, so the operator of equation \ref{eq:BSmat} can be written in the basis of the
Ising axis as
\begin{align*}
    \va{B}\cdot \va{S} &= \pmqty{
            0 & 0 & 0 & 0\\
            0 & -\cos(\vartheta) & \frac{e^{-i\varphi}}{\sqrt{2} }\sin(\vartheta) & 0\\
                    0 & \frac{e^{i\varphi}}{\sqrt{2} }\sin(\vartheta) & 0 &
                    \frac{e^{-i\varphi}}{\sqrt{2} }\sin(\vartheta)\\
                    0 & 0 & \frac{e^{i\varphi}}{\sqrt{2} }\sin(\vartheta) & \cos(\vartheta)
            }
.\end{align*}
Since the Ising contribution to the potential energy is already written in the basis of the
Ising axis in equation \ref{eq:SSmat} the total Hamiltonian arising from potential energy
takes the form:
\begin{align}\label{eq:Hf}
        \mathbb{\cal H}_f &= J\hbar{}^2 \pmqty{
                -1 & 0 & 0 & 0\\
                0 & 1 - \xi \cos(\vartheta)& \frac{\xi e^{-i\varphi}}{\sqrt{2}
                }\sin(\vartheta) & 0\\
                0 & \frac{\xi e^{i\varphi}}{\sqrt{2} }\sin(\vartheta) & -1 & \frac{\xi
                e^{-i\varphi}}{\sqrt{2} }\sin(\vartheta)\\
                0 & 0 & \frac{\xi e^{i\varphi}}{\sqrt{2} }\sin(\vartheta) & 1 + \xi\cos(\vartheta)
        }
.\end{align}
Here \(\xi = -\frac{\gamma B}{J\hbar{}}\) is a proportionality factor between the spin-spin
and spin-field interactions which will simplify calculations. This together with the
kinetic part of the Hamiltonian \(\mathbb{\cal H}_s\) will determine the time evolution of
the system:
\begin{align}\label{eq:Hs}
        \mathbb{\cal H}_s &= \sum_{n=1}^{5} \frac{p_i^2}{2m_i}
.\end{align}
In equation \ref{eq:Hf} it is readily visible that the spin singlet state \(\ket{0, 0}\) is unaffected by
the external magnetic field, as could be concluded even without the explicit Hamiltonian.
As a result the total Hamiltonian for the singlet state is but the sum of two terms
dependent on different sets of variables. This is to say that separation of variables can
be used to solve the eigenstate problem, so the qualities presently at interest are lost.
For this reason henceforth only the non-singlet (so called triplet) states are considered,
and matrices will often be reduced to the relevant three-dimensional subspace for
simplicity's sake.
\section{The Born-Oppenheimer approximation}
\subsection{Derivation}
Solving for the eigenstates of such a Hamiltonian as described above is a mighty task. Note
in particular that the contribution from potential energy to the Hamiltonian, equation
\ref{eq:Hf}, is heavily dependent on the position and rotation of the dumbbell through all
parameters \(\vartheta_r \), \(\vartheta_B\), \(\varphi_r\),
\(\varphi_B\) and \(B\). This couples all degrees of freedom for the system, which complicates
the problem greatly.

An approximation is therefore in order. If the position and rotation,
henceforth the ''slow'' parameters,
is more or less static in comparison with the spin degrees of freedom, henceforth ''fast
parameters'', the so-called Born-Oppenheimer approximation is applicable. A version of the
adiabatic approximation, it assumes that a ''fast'' subsystem, described by the fast
parameters, can be described by eigenstates to a Hamiltonian parametrised by the slow
parameters. A fast system in such an eigenstate can be considered to remain in the same
eigenstate as the associated ''fast'' Hamiltonian slowly changes, changing but its
eigenvalue as the slow parameters evolve. So far this is what is known as the adiabatic
approximation. 

The Born-Oppenheimer approximation involves the extension of this to also consider how the
''slow'' system evolves, in practice finding an effective Hamiltonian to the slow system as
well. The full system is then considered to be described by the product of a wave function to the slow
system \(\Psi_s\) and  some eigenstate to the fast Hamiltonian \(\ket{n}\), i.e.:
\begin{align*}
        \ket{\Psi_{full}} = \Psi_s\ket{n}
.\end{align*}
The aforementioned fast and slow Hamiltonians are for the system in consideration the
previously found \(\mathbb{\cal H}_f\) and \(\mathbb{\cal H}_s\) respectively. The full
solution to the fast system is assumed to be found, i.e.:
\begin{align*}
        \mathbb{\cal H}_f \ket{n} = E_n\ket{n}
.\end{align*}
The Schrödinger equation then implies, since \(\frac{\partial }{\partial t} \ket{n} = 0\):
\begin{align*}
        &i\hbar{}\frac{\partial }{\partial t} (\Psi_s\ket{n}) = (\mathbb{\cal H}_f +
        \mathbb{\cal H}_s) \Psi_s\ket{n}\\
        &i\hbar{}\frac{\partial \Psi_s}{\partial t} = \bra{n}(\mathbb{\cal H}_f +
        \mathbb{\cal H}_s)\ket{n}\Psi_s = (\bra{n}\mathbb{\cal H}_s\ket{n} +
        E_n)\Psi_s
.\end{align*}
This can be interpreted as an effective Hamiltonian \(\mathbb{\cal H}_{eff} =
\bra{n}\mathbb{\cal H}_s\ket{n} + E_n\) governing the slow wave function.
The inner product term can be further manipulated as follows:
\begin{align*}
        \bra{n} \mathbb{\cal H}_s \ket{n} \Psi_s &= \sum_{i=1}^{5} \bigg[\bra{n}
        \frac{p_i^2}{2m_i}\ket{n} \Psi_s + \bra{n} \frac{p_i}{m_i} \ket{n}
        p\Psi_s + p_i^2\Psi\bigg]
.\end{align*}
Here it is to be understood that the momentum operators in \(\mathbb{\cal H}_s\) act on
\textit{both} the spin ket and
and the slow wave fucntion. An operator to the left of a ket and wave function product will
however be understood to act on the ket only, if no clarifying parentheses are written out
explicitly.
Since all \(p_i\) are hermitian operators and can thus be acted on bras to the left
without conjugation the most troublesome term can also be
written as: %Fult skrivsätt? Byt till derivator redan här?
\begin{align*}
        \bra{n} p_i^2\ket{n} &= \bra{p_i n} \ket{p_i n}\\
                                                 &=
                                                 \bra{p_i
                                                 n}\ket{n}\bra{n}\ket{p_i n}
                                                 + \bra{p_i n} (\mathbb{1} - \ket{n}
                                                 \bra{n}) \ket{p_i n}\\
                                                 &= \bra{n}\ket{p_i n}^2 +
                                                         \bra{p_i n} (\mathbb{ 1 } -
                                                         \ket{n}\bra{n})\ket{p_i
                                                         n}
.\end{align*}
An identity relation was used in the second step, where \(\mathbb{1}\) is the identity
operator.
Inserting the derivative form of the momentum operator as seen in section \ref{def:mom}
and rearranging terms with some convenient notation we arrive to the Hamiltonian providing the interesting properties
sought after.

\begin{align}\label{eq:Heff}
        \mathbb{\cal H}_{eff} &= \sum_{i=5}^{5} \frac{(p_i - A_i)^2}{2m_i} + \Phi + E_n\\
        A_i &= i\hbar{}\bra{n}\ket{\partial_i n}\\
        \Phi &= \sum_{i=1}^{5} \frac{\hbar{}^{2}}{2m_i}\bra{\partial_i n}(\mathbb{1} - \ket{n}
    \bra{n})\ket{\partial_i n}
.\end{align}
\subsection{Interpretation}
[Något om magnetfältskopplingen, monopolerna, \(\va{A} = \sum_{i=1}^{5} A_i\), \(\va{p} =
\sum_{i=1}^{5} p_i\). En första blick på att det är här den fina fysiken händer]
\subsection{Dynamics}
Having found an effective Hamiltonian to the slow system the application of this
Hamiltonian to the dynamics of the system remains to be performed. One could proceed with
the quantum mechanical methods applied so far, solving for eigenstates to \(\mathbb{\cal
H}_{eff}\). This however requires finding the dependence of the eigenstates to
\(\mathbb{\cal H}_f\) on the slow parameters, which may not be achievable analytically.
%Säker på att just detta handlar om valet klassisk/kvant?
Instead the slow system can be considered to be effectively lying in the classical domain,
and the Hamiltonian derived by quantum mechanical means will be utilized in the role of the
Hamiltonian for classical mechanics. %motivera varför detta är rimligt

Hamilton's canonical equations indicate the time evolution of \(\va{r}\) :%ref Goldstein
\begin{align*} %Fixa massan i nämnarna 
        \frac{\partial \va{r}}{\partial t} &= \frac{\partial \mathbb{\cal H}_{eff}}{\partial \va{p}} = \frac{\va{p} -
 \va{A}}{m}\\ %%Inför Hadamard-notation?
                \frac{\partial \va{p}}{\partial t}  &= -\frac{\partial \mathbb{\cal H}_{eff}}{\partial \va{r}}=
                \pqty{\frac{\partial
 \va{A}}{\partial \va{r}}}^T \frac{\va{p}-\va{A}}{m} -\frac{\partial \Phi}{\partial \va{r}}
 -\frac{\partial E_n}{\partial \va{r}} = \pqty{\frac{\partial
 \va{A}}{\partial \va{r}}}^T \frac{\partial \va{r}}{\partial t} - \frac{\partial
\Phi}{\partial \va{r}} - \frac{\partial E_n}{\partial \va{r}}  
.\end{align*}
Note in particular that the first of these equations imply that the canonical momentum \(\va{p}\) is
\textit{not} \(m \frac{\partial \va{r}}{\partial t} \). The effective force acting on the
system can be found, utilizing that the synthetic vector potential does not depend explicitly
on time, i.e. that \(\frac{\partial \va{A}}{\partial t}  = 0\):
\begin{align}\label{eq:dyn1}
        m \frac{\partial^2 \va{r}}{\partial t^2} &= \frac{\partial \va{p}}{\partial
        t} - \frac{\partial \va{A}}{\partial t}  = \pqty{\frac{\partial \va{A}}{\partial \va{r}}}^T \frac{\partial
\va{r}}{\partial t} - \pqty{\frac{\partial \va{r}}{\partial t}  \vdot \grad}\va{A}
-\frac{\partial \Phi}{\partial \va{r}} - \frac{\partial E_n}{\partial \va{r}}  
.\end{align}
The Jacobian matrix can be treated elementwise, as well as the second term: 
\begin{align*}
        \frac{1}{i\hbar{}}\pqty{\frac{\partial \va{A}}{\partial \va{r}}}_{ji} &= \partial_i
        \bra{n}\ket{\partial_j n} = \bra{\partial_i n}\ket{\partial_j n} +
        \bra{n}\ket{\partial_i \partial_j n}\\
        \frac{1}{i\hbar{}}\pqty{\pqty{\frac{\partial \va{r}}{\partial t} \vdot \grad}
\va{A}}_i &= \sum_{j=1}^{5} \frac{\partial r_j}{\partial t} \partial_j
                \bra{n}\ket{\partial_i n} =\sum_{j=1}^{5}  \frac{\partial r_j}{\partial
                        t}\pqty{ \bra{\partial_j
                n}\ket{\partial_i n} + \bra{n}\ket{\partial_j \partial_i n}}
.\end{align*}
Insertion into equation \ref{eq:dyn1} then yields a higher dimensional analogue to [the usual
case] for the forces due to the synthetic magnetic field:
\begin{align*}
             \frac{1}{i\hbar{}}\bqty{\pqty{\frac{\partial \va{A}}{\partial \va{r}}}^T \frac{\partial
\va{r}}{\partial t} - \pqty{\frac{\partial \va{r}}{\partial t}  \vdot \grad}\va{A}}_i
 &= \frac{1}{i\hbar{}}F_{i}^A = \sum_{j=1}^{5} \frac{\partial
        r_j}{\partial t} \bqty{\bra{\partial_i n}\ket{\partial_j n}-\bra{\partial_j n}\ket{\partial_i
        n}}\\
        &= \sum_{j=1}^{5} \sum_{l} \frac{\partial r_j}{\partial t}\bqty{\bra{\partial_i
                n}\ket{l}\bra{l}\ket{\partial_j n}- \bra{\partial_j
        n}\ket{l}\bra{l}\ket{\partial_i n}}\\
        &= \sum_{j=1}^{5} \sum_{l \ne n} \frac{\partial r_j}{\partial t}\bqty{\bra{\partial_i
                n}\ket{l}\bra{l}\ket{\partial_j n}- \bra{\partial_j
        n}\ket{l}\bra{l}\ket{\partial_i n}}
.\end{align*}
Here \(\ket{l}\) simply denotes an eigenstate to \(\mathbb{\cal H}_{f}\) of some index
\(l\), and the sum over \(l\) is over all available states. The exclusion of \(n=l\)-terms follows as
\(\bra{\partial_i n}\ket{n}\) is pure imaginary, which can be seen from differentiating
\(\bra{n}\ket{n}\). This rearrangement is highly desirable, for it now so happens that this
allows us to take derivatives of the Hamiltonian instead of the rather tricky
differentiation of the eigenkets. Differentiating the Schrödinger equation and acting on it
with some other eigenbra yields:
\begin{align*}
        \mathbb{\cal H}_f \ket{n} &= E_n \ket{n}\\
        \partial \mathbb{\cal H}_f \ket{n} + H_f \ket{\partial n} &= E_n \ket{\partial n}\\
        \bra{l}\partial \mathbb{\cal H}_f \ket{n} &= \bra{l}\ket{\partial n}(E_n - E_l)
.\end{align*}
Rearranging, a very useful relation emerges:
\begin{align}\label{eq:Hdiff}
        \bra{l}\ket{\partial n} &= \frac{\bra{l}\partial \mathbb{\cal H}_f \ket{n}}{E_n
        - E_l}
.\end{align}
This we can insert into above:
\begin{align}\label{eq:FA}
\frac{1}{i\hbar{}}F_i^A &= \sum_{j=1}^{5} \sum_{l \ne n} \frac{\frac{\partial r_j}{\partial
        t}}{\pqty{E_n-E_l}^2}\bqty{\bra{n}\partial_i \mathbb{\cal H}_f
        \ket{l}\bra{l}\partial_j\mathbb{\cal H}_f\ket{n} - \bra{n}\partial_j \mathbb{\cal
        H}_f\ket{l}\bra{l}\partial_i \mathbb{\cal H}_f\ket{n}}\\
        &= 2\sum_{j=1}^{5} \sum_{l \ne n} \frac{\frac{\partial r_j}{\partial
        t}}{\pqty{E_n-E_l}^2} \Im \bqty{\bra{n}\partial_i \mathbb{\cal H}_f
        \ket{l}\bra{l}\partial_j\mathbb{\cal H}_f\ket{n}}
.\end{align}
Equation \ref{eq:Hdiff} can be used for something similar when evaluating the synthetic
electric potential:
\begin{align}\label{eq:ElPot}
        \Phi &= \sum_{i=1}^{5} \sum_{l \ne n} \frac{\hbar{}^2}{2m_i}\pqty{\bra{\partial_i
    n}\ket{l}\bra{l}\ket{\partial_i n}} = \sum_{i=1}^{5} \sum_{l \ne n}
    \frac{\hbar{}^2}{2m_i}\frac{\bra{n}\partial_i \mathbb{\cal H}_f\ket{l}\bra{l}\partial_i
    \mathbb{\cal H}_f\ket{n}}{\pqty{E_n-E_l}^2}\\
    &= \frac{\hbar{}^2}{2m}\sum_{l \ne n}^{} \frac{\bra{n}\grad \mathbb{\cal H}_f
    \ket{l}\vdot \bra{l}\grad\mathbb{\cal H}_f\ket{n}}{\pqty{E_n-E_l}^2}
.\end{align}
Note however that no simple form to the derivative of the electric potential \(\Phi\) has
been found, which might not be easily described analytically.

The problem has thus been reduced to evaluating, per equations \ref{eq:FA} and
\ref{eq:ElPot},
\begin{align}\label{eq:dynfin}
        m\frac{\partial^2 \va{r}}{\partial t^2} &= \va{F}^A-\frac{\partial \Phi}{\partial
        \va{r}} - \frac{\partial E_n}{\partial \va{r}} 
.\end{align}
\subsection{Differentiation of the Hamiltonian}\label{sec:Hdiff}
In order to easily evaluate equations \ref{eq:FA} and \ref{eq:ElPot} derivatives of
\(\mathbb{\cal H}_f\) are to be found. Writing any of the coordinates \(x\), \(y\), \(z\)
as \(r\), and remembering that \(\vartheta=\vartheta_B(r) -
\vartheta_r\), \(\varphi=\varphi_B(r) - \varphi_r\), \(\xi = \xi(r)\) the derivatives can
be written:
\begin{align}
        \frac{1}{J\hbar{}^2}  \partial_r\mathbb{\cal H}_f  &= \pmqty{
 -\dot \xi \cos(\vartheta) + \xi \dot \vartheta_B \sin(\vartheta) & \Omega & 0 \\
 \overline{\Omega} & 0 & \Omega\\
 0 & \overline{\Omega}& \dot \xi \cos(\vartheta)- \xi \dot \vartheta_B
\sin(\vartheta)
        }\\
        \frac{1}{J\hbar{}^2}\partial_{\vartheta_r} \mathbb{\cal H}_f &= \pmqty{
                 -\xi\sin(\vartheta) &
                -\frac{\xi e^{-i\varphi}}{\sqrt{2}}\cos(\vartheta)&0 \\
                -\frac{\xi e^{i\varphi}}{\sqrt{2}}\cos(\vartheta)& 0 &
                -\frac{\xi e^{-i\varphi}}{\sqrt{2}}\cos(\vartheta)\\
                0 & -\frac{\xi e^{i\varphi}}{\sqrt{2}}\cos(\vartheta) & \xi
                \sin(\vartheta)
        }\\
        \frac{1}{J\hbar{}^2}\partial_{\varphi_r} \mathbb{\cal H}_f &= \pmqty{
                 0 & -i\frac{\xi e^{-i\varphi}}{\sqrt{2}}\sin(\vartheta) & 0\\
                 i\frac{\xi e^{i\varphi}}{\sqrt{2}}\sin(\vartheta) & 0 & -i\frac{\xi
                e^{-i\varphi}}{\sqrt{2}}\sin(\vartheta)\\
                 0 & i\frac{\xi e^{i\varphi}}{\sqrt{2}}\sin(\vartheta) & 0
        }
.\end{align}
Here \(\Omega = \frac{\dot \xi
e^{-i\varphi}}{\sqrt{2} }\sin(\vartheta) - i\dot \varphi_B \frac{\xi
e^{-i\varphi}}{\sqrt{2}}\sin(\vartheta) + \frac{\xi \dot \vartheta_B
e^{-i\varphi}}{\sqrt{2}}\cos(\vartheta)\) is introduced as a means of compressing the
rather lengthy expressions for the derivative with respect to \(r\).
\subsection{Solution of the fast subsystem}
The usage of the Born-Oppenheimer approximation requires a solution to the fast subsystem,
i.e. that the eigenvalues and eigenvectors to \(\mathbb{\cal H}_f\) are found.
Unfortunately this is not possible analytically for the present system, but we note that it
is the same as solving the following transcendental characteristic equation, which follows
from equation \ref{eq:Hf}, for the
eigenvalues \(\lambda_n = \frac{E_n}{J\hbar{}^2}\): %Citera Sjöqvist, Yi
\begin{align}
        0 &= \mdet{
                \lambda_n - 1 + \xi \cos(\vartheta) & -\frac{\xi e^{-i\varphi}}{\sqrt{2}
                }\sin(\vartheta) & 0 \\
                -\frac{\xi e^{i\varphi}}{\sqrt{2}
                }\sin(\vartheta) & \lambda_n + 1 & -\frac{\xi e^{-i\varphi}}{\sqrt{2}
                }\sin(\vartheta)\\
                0 & -\frac{\xi e^{i\varphi}}{\sqrt{2}
                }\sin(\vartheta) & \lambda_n - 1 - \xi\cos(\vartheta)}\\
                &=
                 \lambda_n^3 - \lambda_n^2 -(1+ \xi^2)\lambda_n + 1 -
                 \xi^2\cos(2\vartheta)\label{eq:lambda}
.\end{align}

Once a numerical solution for \(\lambda_n\) has been performed the corresponding eigenvectors \(\ket{n}\) can be
found by the following formula:
\begin{align*}
      \ket{n} &= a\ket{1, -1} + \ket{1, 0} + b\ket{1, 1} \\
      a &= \frac{\xi e^{-i\varphi}\sin(\vartheta)}{\lambda_n -1 + \xi\cos(\vartheta)} \\
      b &= \frac{\xi e^{i \varphi}\sin(\vartheta)}{\lambda_n - 1 -\xi\cos(\vartheta)}
.\end{align*}

Since derivatives of \(E_n\) are of interest to the dynamics in equation \ref{eq:dynfin}
it is also of interest to perform implicit differentiation on equation \ref{eq:lambda}. The
results are as follows, using the same notation as section \ref{sec:Hdiff}.
\begin{align*}
        \partial_r E_n &= 2J\hbar{}^2 \frac{\dot \xi \xi \cos(2\vartheta) - \dot \vartheta \xi^2
    \sin(2\vartheta)}{3\lambda_n^2 - 2\lambda_n - 1 - \xi^2}\\
    \partial_{\vartheta_r} E_n &= 2J\hbar{}^2 \frac{\xi^2\sin(2\vartheta)}{3\lambda_n -
    2\lambda_n -1 -\xi^2} \\
            \partial_{\varphi_r} E_n &= 0 %%Va
.\end{align*}

\end{document}
