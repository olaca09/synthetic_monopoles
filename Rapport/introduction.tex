\documentclass[main.tex]{subfiles}

\begin{document}
\section{Introduction}
%\subsection{On the occurrence of monopoles}
\label{sec:introduction}
Our theory of electromagnetism carries certain asymmetries between the magnetic and
electric fields. A well known and often discussed such distinction is the absence of
magnetic monopoles, i.e., that
any analogue of electric charge is absent for the magnetic field. It is an extension often
suggested by theorists, initially by Dirac in 1931 \cite{dirac}, to include magnetic
charges and also potentially magnetic currents in Maxwell's equations, but
there is as of yet no accepted empirical data to support this cause. Magnetic monopoles do
however appear occasionally in a less fundamental sense, as emergent phenomena in
many-body systems for example in
spin ice \cite{castelnovo} or Bose-Einstein condensates \cite{ray}. 

%Specifiera adiabaticitet redan här?
Another area in which magnetic monopoles has appeared is the study of \textit{geometric}
phases of quantum systems, first described by Berry in 1984 \cite{berry1984}. Roughly
speaking this is the phase acquired by the wave function of a system which is not dynamic
in origin. The dynamical phase of a system state is induced by the energy of that state,
while the geometric phase is surprisingly enough \textit{in}dependent of the energy values,
and rather arises as a function of the path taken by the system through the relevant
''parameter-space'' parametrising the Hamiltonian. For certain systems this geometric phase evolution
resembles the %dynamical ??
evolution of a charged system in a magnetic vector potential field that lives in 
parameter space. This field, here named the \textit{synthetic} magnetic field, happens to
exhibit non-zero divergence at certain points, the degeneracy points of the state energies,
which thus correspond to monopoles of the synthetic field.

Geometrical phase is an interesting field of study in itself, but it is with the
appearance of magnetic-type monopoles that the present body of work finds its premise.
While the study of fundamental magnetic monopoles remains impossible, we can through construction of
a suitable system study the effects of magnetic monopoles through their action on the
system state in parameter space. Parameter space can be put into correspondence with real
space, and the movement of charged matter through monopole fields
becomes measurable, even though fundamental monopoles remain fictitious.

%\subsection{A select system}
\label{sec:aselsys}
Berry's original 1984 article considers a simple spin-system with a single magnetic dipole
moment in an external magnetic field \(\va{B}\). The relevant parameter space is the space of all
possible external fields, and the geometric phase contribution takes the form of a
synthetic magnetic field purely generated by a monopole sitting at the origin, i.e., at
\(\va{B} = \va{0}\). This field is the simplest example of a monopolar field,
so to find more complex behaviour this starting point of a system can be extended to
include multiple spin components, i.e., multiple magnetic dipoles, and interactions between
those dipoles. The effects of introducing such interactions is roughly that of splitting
the origin-centred monopole into smaller constituent parts whose positions in parameter space
depend on the exact nature of the spin-spin interactions \cite{eriksson}.

This splitting is desired, and so the system studied will be composed of two massive
spin-\(\frac{1}{2}\) components that interact through the so-called Ising interaction
described in section \ref{sec:sysham}. This is to some extent the simplest spin-spin
interaction and is dependent only on spin along a chosen axis, here taken to be the axis
connecting the two masses. Movement through
parameter space can be mapped to movement of the center of mass through real space given an external
inhomogeneous magnetic field, and the movement of the spin components relative to one another is as
a simplest case the rotation through polar and azimuthal angles with fixed inter-component
distance. Such a setup is reminiscent of a dumbbell translating and rotating through space,
with the added complication that each ''weight'' of the dumbbell acts as a dipole (has
spin) interacting with both the dipole at the other weight and an external magnetic field,
see figure \ref{fig:dumbbell}.

\begin{figure}[h]
        \centering
        \begin{tikzpicture}
            \draw[black] node[anchor=north, xshift=-0.2cm]{\Huge\(\ket{s, m_1}\)} (0,0) -- (4.9,3.4)
                    node[anchor=south, xshift=0.4cm]{\(\ket{s, m_2}\)};
            \draw[black] (1, -0.3) -- (5,3.4);
            \draw[blue, postaction=decorate, decoration={pre length = 1mm, post length = 1mm, markings, mark=
                        between positions 0.1 and 1 step
                        0.25 with {\pgftransformscale{2}\arrow{stealth}}}] plot [smooth, tension=1] coordinates {(-1,3.5) (1,1) (4,0)};
            \draw[blue, postaction=decorate, decoration={pre length = 1mm, post length = 1mm, markings, mark=
                        between positions 0.1 and 1 step
                        0.4 with {\pgftransformscale{2}\arrow{stealth}}}] plot [smooth, tension=1] coordinates {(3,3.7) (3.5,2.5) (4.3,1.5)};
            \draw[blue, postaction=decorate, decoration={pre length = 1mm, post length = 1mm, markings, mark=
                        between positions 0.1 and 1 step
                        0.28 with {\pgftransformscale{2}\arrow{stealth}}}] plot [smooth, tension=1] coordinates {(1,3.6) (2.5,1.5) (4.2,0.8)};
            \draw[blue] (4.2,0.8) node[anchor=west]{\Huge\(\va{B}\) };
        \end{tikzpicture}
        \caption{A dumbbell with two spin components moving through a magnetic field.}
        \label{fig:dumbbell}
\end{figure}

It is the time evolution of this system, henceforth referred to as the dumbbell, with which
this project is concerned. Approximate equations of motion will be derived and then be put
to test in a numerical simulation. Underlying the process is the hope of discerning effects of the synthetic
magnetic monopoles on the movement of the dumbbell.
%\subsection{Interpretations of the system}
In addition to carrying dynamics of interest, the described dumbbell is also appropriate
for its potential to model realisable physical systems. Such a realisation of the model
herein described opens up the possibility of experimentally measuring the action of the
synthetic fields, and by extension the action of synthetic monopoles.

Two paths of realisation spring to mind: Firstly a diatomic molecule with appropriate effective
spin of the two constituent atoms could be tested. It would be of importance that the
magnetic moment of each atom be large, so that the molecule couples to the external
field strongly enough, and it would further be desired that spin-spin interactions
between the molecules be strong so as to achieve the more exotic synthetic field texture
mentioned in section \ref{sec:aselsys}. Preferably the gas phase of such a molecule should
be obtainable, for it would be desired to measure the dynamics of single such
molecules without inter-molecular interaction. For the dumbbell model to apply reasonably
well the interatomic binding would also have to be of such a nature that the interatomic
distance would not vary greatly. The plausibility of this approach, and the selection of
suitable candidate molecules, is an interesting question in its own right and warrants
further research. Something as simple as hydrogen gas is not necessarily unsuited.

Secondly, one might imagine a substantially smaller system consisting of a single atom with
non-negligible nuclear and electronic spin. The same considerations concerning the strength
of interactions apply as above, but this method appears to carry
larger obstructions. Nuclear magnetic moments, while measurable due to resonance effects such as in
nuclear magnetic resonance (NMR) measurements, are orders of magnitude smaller than their
electronic counterparts \cite{krane}. This, together with the disparate masses of electrons
and nucleus would necessitate a heavily asymmetric dumbbell model. It is also a well known
fact that such classical approximations as the definite position of the electronic part of
the system implied by a dumbbell model break down at these length scales. We must consider
an atom a quantum thing, for if we do not we will find incorrect results.
\end{document}
