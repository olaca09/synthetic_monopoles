\documentclass[a4paper]{article}

\usepackage[utf8]{inputenc}
\usepackage[T1]{fontenc}
\usepackage{textcomp}
\usepackage[swedish]{babel}
\usepackage{amsmath, amssymb}
\usepackage[arrowdel]{physics}
\usepackage{pdfpages}
\usepackage{graphicx}
\usepackage{hyperref}
\usepackage{appendix}
\usepackage[super, square]{natbib}
\usepackage{listings}
\usepackage[section]{placeins}

\usepackage{tikz}

\newcommand{\R}{\ensuremath{\mathbb{R}}}
\newcommand{\C}{\ensuremath{\mathbb{C}}}
\newcommand{\Arg}{\ensuremath{\text{Arg}}}

\usetikzlibrary{arrows.meta}
\usetikzlibrary{decorations.pathreplacing}
\usetikzlibrary{decorations.markings}
\usetikzlibrary{patterns}
\lstset{language=Python, breaklines, literate=%
{å}{{\r{a}}}1
{ä}{{"a}}1
{ö}{{"o}}1
{Å}{{\r{A}}}1
{Ä}{{"A}}1
{Ö}{{"O}}1
}

\pdfsuppresswarningpagegroup=1

\begin{document}
\section{Introduction}
\subsection{On the occurrence of monopoles}
Our theory of electromagnetism carries certain asymmetries between the magnetic and electric fields. Such a distinction well known and often discussed is the absence of magnetic monopoles, i.e. that
any analogue of electric charge is absent for the magnetic field. It is an extension commonly suggested by theorists to include magnetic charges, and therefore currents, in Maxwell's equations, but
there is as of yet no accepted empirical data to support this cause. Magnetic monopoles do however appear every now and then in a less fundamental sense, as for example 
\end{document}
