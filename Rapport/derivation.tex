\documentclass[a4paper]{article}

\begin{document}
\section{System description}
\subsection{Coordinates and quantities}
Here follows a complete model of the scenario outlined in section \ref{sec:aselsys}, with
the purpose of simulating the system numerically to gain insights into the dynamics of
synthetic magnetic fields. Consider a dumbbell-like system consisting of two equal masses at a distance \(l\) from one
another, and let \(m\) be the total mass. The system can be freely translated and rotated
throughout space, so let \(x\), \(y\), \(z\)  be the position
of the centre off mass, and \(\theta_r\), \(\varphi_r\) be the polar and azimuthal angle
respectively of the axis connecting the two masses. Notate these coordinates compactly as the
vector \[
\va{r}
= \begin{pmatrix} x\\ y \\ z\\ \vartheta_r \\\varphi_r\end{pmatrix}
.\] 
Fix the angles such that a polar angle of \(\vartheta_r = 0\) implies a dumbbell parallel
to the \(z\)-axis and so that an azimuthal angle of \(\varphi_r = 0\) implies that the dumbbell
axis lies in the \(xz\)-plane. 

Consider also each of the masses of the dumbbell to carry spin, intrinsic angular
momentum, of size \(\frac{1}{2}\) each. %more precise formulation?
 The state of the spin components must be
described quantum mechanically, so let \(\ket{s, m'}\) denote the state of the system with
\textit{total} spin magnitude squared \(s(s+1)\hbar^2\) and \textit{total} spin measured
along the \(z\)-axis \(\hbar m'\). Note that the spin quantum number \(s\) will be \(1\) for
the composite system, and so values of \(m'\) will range from \(-1\) to \(1\). An external field \(\va{B}\) is present, which
we can describe by its magnitude \(B\) and its angular direction \(\vartheta_B, \varphi_B\)
in analogue with the angles defined above.

\subsection{The Hamiltonian}\label{sec:sysham}

The time evolution of such a system is governed in both classical and quantum mechanics by
its Hamiltonian. Since spin is the epitome of a phenomena demanding a quantum mechanical
interpretation we have no choice but to model the whole system quantum mechanically. %, at
%least for now?
The Hamiltonian which will be assumed for the system is:

\begin{equation}\label{eq:Hamil0}
        \mathbb{\cal H} = \sum_{i=1}^{5} \frac{\va{p_i}^2}{2m_i} +
        \frac{4J}{\hbar{}}S^{(1)}_{\mu}S^{(2)}_{\mu} -
        \gamma\va{B}(\va{r})\cdot \va{S}.
\end{equation}

The first sum is over the five degrees of translational and rotational coordinates in \(\va{r}\)
. Their conjugate momentum operators are
taken to be \(p_i = i\hbar \partial_i\label{def:mom}\) with \(\partial_i\) as the derivative with respect
to the
corresponding coordinate. Note that it is not a priori clear that the effective masses
\(m_i\) for
all degrees of freedom are the same, but we can until later note that at least the first
three are equal to \(m\).

For the potential energy the spin-spin interaction is taken to be of Ising form, which is
the first term after the sum, while the Zeeman interaction between spin and magnetic field is
considered in the final term. %correct name of the final term?
An Ising interaction requires a preferred axis, which for symmetry reasons
of the system has been chosen to be the direction \(\mu\) of the dumbbell axis, the axis connecting
the two masses. The necessity for selecting an axis is the reason for choosing
precisely an Ising interaction, as it breaks the spherical symmetry of the system and
allows for more exotic synthetic field textures as described in section \ref{sec:regmono}. The parameters \(J\) and \(\gamma\) are the strengths of both
of these interactions, while the operators \(\va{S}\) and \(S^{(n)}_{\mu}\) are respectively
the one
related to the total spin of the system and the spin in the \(\mu\)-direction of the \(n\)th system
component. Note that the parameter \(\gamma\) much like the example of section
\ref{sec:adiab} typically looks like \(\gamma = \frac{g\mu_f}{\hbar{}}\), where \(g\) is a
g-factor and \(\mu _f\) is some appropriate magneton.

Similar systems as the one considered here has been studied before in the context of geometric phase, in particular
a bipartite spin-\(\frac{1}{2}\) system with coordinate-fixed Ising axis by Yi and Sjöqvist
\cite{yi}. If the rotational degrees of the present system are ignored and the coordinate
\(\vartheta_r\) is set to \(0\) whenever present the system of \cite{yi} will be
matched in full, save for that the external field is there varied directly instead of
through centre of mass motion.

\subsection{Effective mass}
To clearly see the values of the effective masses paired with the rotational momenta a quick derivation
of the kinetic part of the Hamiltonian is in order. The kinetic energy related to rotation
is of the form \[
K_{rot}= \frac{m}{2}\pqty{\frac{l}{2}}^2\pqty{\dot \vartheta_r^2 + \dot \varphi_r^2} % fac
%2 fel
,\] 
which is the same as the relevant terms of the Lagrangian. The quantum mechanical momenta
correspond to the momenta received from differentiating the classical Lagrangian, and as
of %fråga Erik
such we have in the classical picture that 
\begin{align*}
        p_4 &= \frac{\partial K}{\partial \dot \vartheta_r} = \frac{ml^2}{4}\dot
        \vartheta_r,\\
        p_5 &= \frac{\partial K}{\partial \dot \varphi_r} = \frac{ml^2}{4}\dot \varphi_r
.\end{align*}
Performing the Legendre transform from the Lagrangian to the Hamiltonian yields:
\begin{align*}
        \mathbb{\cal H}_{rot}= p_4 \dot\vartheta_r + p_5 \dot\varphi_r - K_{rot} = \frac{
        p_4^2 + p_5^2}{2} \frac{4}{ml^2}
.\end{align*}
It is then clear that the effective masses to be used in equation \ref{eq:Hamil0} are:
\begin{align*}
        m_i = \begin{cases}
                m, & i = 1, 2, 3,\\
                \frac{ml^2}{4}, & i = 4, 5.
        \end{cases}
\end{align*}
The rotational effective ''masses'' are of course moments of inertia, but will be
referred to as masses such that all five degrees of freedom are treated equally.

\subsection{Rotation matrices}
The potential energy operators will be of great use in some matrix form, so let the spin
state of the entire system be described in the total spin basis \((\ket{0, 0}, \ket{1, -1}, \ket{1, 0},
\ket{1, 1})\) with the coordinate \(z\)-axis as the spin measurement direction. In the special case where the axis of the dumbbell (henceforth ''Ising
axis'') and the magnetic field \(\va{B}\) are parallel to the \(z\)-axis, it is clear that the operators take the form:
\begin{align}
        \va{B}\cdot \va{S} &= B\hbar \label{eq:BSmat}
        \begin{pmatrix}
        \dmat{0,-1,0,1}
        \end{pmatrix} \\
        \text{and}\nonumber\\
        S^{(1)}_{\mu}S^{(2)}_{\mu} &= \frac{\hbar^2}{4} \label{eq:SSmat}
        \begin{pmatrix} 
        \diagonalmatrix{-1,1,-1,1}\end{pmatrix} 
.\end{align}

The second matrix follows from the well known representation of a two-component
spin-\(\frac{1}{2}\) system as singlet and triplet states: Let \(\ket{m_1}\otimes \ket{m_2}\) be the
state with spin-\(z\) number \(m_1\) for the first spin and \(m_2\) for the second
spin. Then
\begin{align*}
    \ket{0, 0} &= \frac{1}{\sqrt{2}}\bqty{\ket{\frac{1}{2}}\otimes\ket{-\frac{1}{2}} -
                \ket{-\frac{1}{2}}\otimes\ket{\frac{1}{2}}}\\
    \ket{1, -1} &= \ket{-\frac{1}{2}}\otimes\ket{-{\frac{1}{2}}}\\
    \ket{1, 0} &= \frac{1}{\sqrt{2} }\bqty{\ket{\frac{1}{2}}\otimes\ket{-\frac{1}{2}} +
            \ket{-\frac{1}{2}}\otimes\ket{\frac{1}{2}}}\\
    \ket{1, 1} &= \ket{\frac{1}{2}}\otimes\ket{\frac{1}{2}}
.\end{align*}

Both matrices above assume that the basis is aligned with \(\va{B}\) and the Ising axis
respectively. Therefore some rotation operator must be found that can describe a state
given in the \(z\)-axis basis in a basis aligned with \(\va{B}\) or the Ising axis.

Consider therefore first a rotation of the \textit{state} vectors, which can 
be inverted to receive the forward transformation also necessary for the transformation
of operator matrices. The inversion process is but a complex
conjugation since the operator in question is unitary. It can be shown 
that the rotation about three Euler angles \(\alpha\), \(\beta\), \(\delta\) of a state is
given  by the matrix with elements as \cite{sakurai}:
\begin{align*}
        \U_{m'm''} = \bra{s,
        m'}e^{\frac{-iS_\mu\alpha}{\hbar}}e^{\frac{-iS_y\beta}{\hbar}}e^{\frac{-iS_\mu\delta}{\hbar{}}}\ket{s,
m''}
.\end{align*}
Here, \(s\), \(m'\), \(m''\) are spin quantum numbers of the system, which in the more general
case can be replaced by angular momentum quantum numbers. The rotations \(\alpha\),
\(\beta\) and \(\delta\) are done about the \(z-\), \(y-\) and then \(z-\) body axes of
the system. Since the spin states considered here are symmetric about their
body \(z\)-axes the final rotation \(\delta\) is superfluous and thus will be discarded.
Identifying the angles \(\alpha = \varphi\) and \(\beta = \vartheta\) for rotation to some
spherical coordinates it can further be
shown that the exponential operators amount to:
\begin{align*}
        \U = \pmqty{
        1 & 0 & 0 & 0\\
        0 & \frac{e^{-i\varphi}}{2}(1+\cos(\vartheta)) &
        \frac{e^{-i\varphi}}{\sqrt{2}}\sin(\vartheta) &
        \frac{e^{-i\varphi}}{2}(1-\cos(\vartheta))\\
        0 & -\frac{1}{\sqrt{2}}\sin(\vartheta) & \cos(\vartheta) & \frac{1}{\sqrt{2} }
        \sin(\vartheta)\\
        0 & \frac{e^{i\varphi}}{2}(1-\cos(\vartheta)) &
        -\frac{e^{i\varphi}}{\sqrt{2}}\sin(\vartheta) & \frac{e^{i\varphi}}{2}(1 +
        \cos(\vartheta))
}
.\end{align*}
An operator matrix \(A\) transforms under rotation as \(A_{rot} = \U A\U^{\dagger}\), so
the operator of equation \ref{eq:BSmat}, which is expressed in terms of a basis rotated by
angles \(\vartheta_B\) and \(\varphi_B\), can be written in the \(z\)-axis basis as:
\begin{align}\label{eq:BSrot}
    \va{B}\cdot \va{S} &= B\hbar{}\pmqty{
            0 & 0 & 0 & 0\\
            0 & -\cos(\vartheta_B) & \frac{e^{-i\varphi_B}}{\sqrt{2} }\sin(\vartheta_B) & 0\\
                    0 & \frac{e^{i\varphi_B}}{\sqrt{2} }\sin(\vartheta_B) & 0 &
                    \frac{e^{-i\varphi_B}}{\sqrt{2} }\sin(\vartheta_B)\\
                    0 & 0 & \frac{e^{i\varphi_B}}{\sqrt{2} }\sin(\vartheta_B) & \cos(\vartheta_B)
            }
.\end{align}
Analogously, the matrix of equation \ref{eq:SSmat} is expressed in a basis rotated through
angles \(\vartheta_r\) and \(\varphi_r\), so in the \(z\)-axis basis it can be written:
\begin{align}\label{eq:SSrot}
        S^{(1)}_\mu S^{(2)}_\mu &= \frac{\hbar{}^2}{4}\pmqty{-1 & 0 & 0 & 0\\
                0 & \cos[2](\vartheta_r) & -\frac{e^{i\varphi_r}}{\sqrt{2}
                }\sin(2\vartheta_r) & e^{-2i\varphi_r}\sin[2](\vartheta_r)\\
                0 & -\frac{e^{-i\varphi_r}}{\sqrt{2}
                }\sin(2\vartheta_r) & -\cos(2\vartheta_r) & \frac{e^{-i\varphi_r}}{\sqrt{2}
        }\sin(2\vartheta_r)\\
        0 & e^{2i\varphi_r}\sin[2](\vartheta_r) & \frac{e^{i\varphi_r}}{\sqrt{2}
        }\sin(2\vartheta_r) & \cos[2](\vartheta_r)}
.\end{align}
In equations \ref{eq:BSrot} and \ref{eq:SSrot} it is readily visible that the spin singlet state \(\ket{0, 0}\) is unaffected by
the external magnetic field, as could be concluded even without the explicit Hamiltonian.
As a result the total Hamiltonian for the singlet state is but the sum of two terms
dependent on different sets of variables. This is to say that separation of variables can
be used to solve the eigenstate problem, so the qualities presently at interest are lost.
For this reason henceforth only the non-singlet, that is triplet, states are considered,
and matrices will subsequently be reduced to the relevant three-dimensional subspace for
simplicity's sake.

At last all parts of the potential energy are expressed in a single basis, such that the
potential part of the Hamiltonian takes the form:
\begin{align}\label{eq:Hf}
        \mathbb{\cal H}_f &= \gamma B \hbar{} \pmqty{
                 \xi\cos[2](\vartheta_r) + \cos(\vartheta_B)& -\frac{
                        e^{-i\varphi_B}}{\sqrt{2}
                }\sin(\vartheta_B) - \xi\frac{e^{-i\varphi_r}}{\sqrt{2} }\sin(2\vartheta_r) &
                \xi e^{-2i\varphi_r}\sin[2](\vartheta_r)\\
                 -\frac{e^{i\varphi_B}}{\sqrt{2}}\sin(\vartheta_B) -
                \xi \frac{e^{i\varphi_r}}{\sqrt{2} }\sin(2\vartheta_r) &
                -\xi\cos(2\vartheta_r) & -\frac{
                e^{-i\varphi_B}}{\sqrt{2} }\sin(\vartheta_B)+ \xi\frac{e^{-i\varphi_r}}{\sqrt{2}
        }\sin(2\vartheta_r)\\
                \xi e^{2i\varphi_r}\sin[2](\vartheta_r) & -\frac{
                e^{i\varphi_B}}{\sqrt{2} }\sin(\vartheta_B) +
                        \xi \frac{e^{i\varphi_r}}{\sqrt{2} }\sin(2\vartheta_r) &
                        \xi \cos[2](\vartheta_r) - \cos(\vartheta_B) 
        }
.\end{align}
Here, \(\xi = \frac{J}{\gamma B}\) is a proportionality factor between the spin-spin
and spin-field interactions. This together with the
kinetic part of the Hamiltonian
\begin{align}\label{eq:Hs}
        \mathbb{\cal H}_s &= \sum_{n=1}^{5} \frac{p_i^2}{2m_i}
.\end{align}
will determine the time evolution of the system.
\section{The Born-Oppenheimer approximation}
\subsection{Derivation}
Solving for the eigenstates of such a Hamiltonian as described above is a mighty task. Note
in particular that the contribution from potential energy to the Hamiltonian, equation
\ref{eq:Hf}, is heavily dependent on the position and orientation of the dumbbell through all
parameters \(\vartheta_r \), \(\vartheta_B\), \(\varphi_r\),
\(\varphi_B\) and \(B\). This couples all degrees of freedom for the system, which complicates
the problem greatly.

An approximation is therefore in order. If the position and orientation,
henceforth the ''slow'' parameters,
are more or less static in comparison with the spin degrees of freedom, henceforth ''fast''
parameters, the Born-Oppenheimer approximation is applicable. A version of the
adiabatic approximation described in section \ref{sec:adiab}, it assumes that a ''fast''
subsystem, dependent on the fast
parameters, can be described by eigenstates of a Hamiltonian parametrised by the slow
parameters. A fast system in such an eigenstate can be considered to remain in the same
eigenstate as the associated ''fast'' Hamiltonian slowly changes, changing its
eigenvalue as the slow parameters evolve. So far this is just the adiabatic
approximation. In the terminology used for geometric phases the slow
parameters then correspond to the parameter space of the fast system. %Var specifierar jag att jag håller mig inom adiabatiskt område?
%Resultat/diskussion antagligen

The Born-Oppenheimer approximation involves the extension of this to also consider how the
''slow'' system evolves, in practice finding an effective Hamiltonian to the slow system as
well. Originally an approximation used in molecular physics proposed in
1927 \cite{bo}, it applies also to the present situation. The full system is considered to be described by the product of a wave function to the slow
system \(\Psi_s\) and  some eigenstate of the fast Hamiltonian \(\ket{n}\), i.e.,
\begin{align*}
        \ket{\Psi_{full}} = \Psi_s\ket{n}
.\end{align*}
The aforementioned fast and slow Hamiltonians are for the system in consideration the
previously found \(\mathbb{\cal H}_f\) and \(\mathbb{\cal H}_s\), respectively. The full
solution of the fast system is assumed to be known, i.e., that
\begin{align*}
        \mathbb{\cal H}_f \ket{n} = E_n\ket{n}
\end{align*}
is solved.
The Schrödinger equation implies, since \(\frac{\partial }{\partial t} \ket{n} = 0\),
\begin{align*}
        &i\hbar{}\frac{\partial }{\partial t} (\Psi_s\ket{n}) = (\mathbb{\cal H}_f +
        \mathbb{\cal H}_s) \Psi_s\ket{n}\\
        &i\hbar{}\frac{\partial \Psi_s}{\partial t} = \bra{n}(\mathbb{\cal H}_f +
        \mathbb{\cal H}_s)\ket{n}\Psi_s = (\bra{n}\mathbb{\cal H}_s\ket{n} +
        E_n)\Psi_s
.\end{align*}
This can be interpreted as an effective Hamiltonian \(\mathbb{\cal H}^{(n)}_{eff} =
\bra{n}\mathbb{\cal H}_s\ket{n} + E_n\) governing the slow wave function.
The inner product term can be further manipulated in our system as follows:
\begin{align*}
        \bra{n} \mathbb{\cal H}_s \ket{n} \Psi_s &= \sum_{i=1}^{5} \bigg[\bra{n}
        \frac{p_i^2}{2m_i}\ket{n} \Psi_s + \bra{n} \frac{p_i}{m_i} \ket{n}
        p_i\Psi_s + \frac{p_i^2}{2m_i}\Psi_s\bigg]
.\end{align*}
Here, it is to be understood that the momentum operators in \(\mathbb{\cal H}_s\) act on
\textit{both} the spin ket and
and the slow wave function. An operator to the left of a ket and wave function product will
however be understood to act on the ket only, if no clarifying parentheses are written out
explicitly.
Since all \(p_i\) are Hermitian operators and can thus be acted on bras to the left
without conjugation the most troublesome term can also be
written as: %Fult skrivsätt? Byt till derivator redan här?
\begin{align*}
        \bra{n} p_i^2\ket{n} &= \bra{p_i n} \ket{p_i n}\\
                                                 &=
                                                 \bra{p_i
                                                 n}\ket{n}\bra{n}\ket{p_i n}
                                                 + \bra{p_i n} (\mathbb{1} - \ket{n}
                                                 \bra{n}) \ket{p_i n}\\
                                                 &= \bra{n}\ket{p_i n}^2 +
                                                         \bra{p_i n} (\mathbb{ 1 } -
                                                         \ket{n}\bra{n})\ket{p_i
                                                         n}
.\end{align*}
An identity relation was used in the second step, where \(\mathbb{1}\) is the identity
operator.
Inserting the derivative form of the momentum operator as seen in section \ref{def:mom}
and rearranging terms with some convenient notation we arrive to the Hamiltonian providing the interesting properties
sought after:

\begin{align}\label{eq:Heff}
        \mathbb{\cal H}^{(n)}_{eff} &= \sum_{i=5}^{5} \frac{(p_i - A^{(n)}_i)^2}{2m_i} + \Phi^{(n)} + E_n,\\
        A^{(n)}_i &= i\hbar{}\bra{n}\ket{\partial_i n},\\
        \Phi^{(n)} &= \sum_{i=1}^{5} \frac{\hbar{}^{2}}{2m_i}\bra{\partial_i n}(\mathbb{1} - \ket{n}
    \bra{n})\ket{\partial_i n}
.\end{align}
\subsection{Interpretation}\label{sec:BOinterp}
Equation \ref{eq:Heff} has been aptly written on a form which suggests the physics to be
studied. Note that the sum over \(i\) looks precisely like the Hamiltonian of a magnetic field
with vector potential \(\va{A}^{(n)} = i\hbar{}\bra{n}\ket{\grad n}\) on a particle of charge
\(1\) and momentum \(\va{p} = (p_1, \ldots,p_5)\). Note also that this magnetic field is
the same as the synthetic magnetic field outlined in section \ref{sec:geophase}, and as of
such carries precisely the same properties. In particular the field will carry a monopolar
dependence, as desired. A difference present to Maxwellian magnetic fields is that both field and momentum are here five dimensional, and
furthermore that the masses of the two final degrees of freedom are rather moments of
inertia. The dynamics of this term is the
main interest of this discussion, but we note also a scalar field \(\Phi^{(n)}\), analogously called the
synthetic electric field or the synthetic scalar field. Roughly speaking however this
field is a factor \(\hbar{}\) smaller than the synthetic magnetic field and will in most
cases be negligible. 

It can however be shown that the scalar field acts as a repulsive inverse
square force near degeneracies in the fast Hamiltonian\cite{berrylim}. The inverse square dependence to
the distance of a degeneracy point means that the scalar field will have appreciable
effects if the slow parameters are close enough to the degeneracy, and furthermore the
repulsive nature actually leads to a strengthening of the Born-Oppenheimer approximation as
the adiabatic approximation loses validity at points of degeneracy.
\subsection{Dynamics}
Having found an effective Hamiltonian to the slow system the application of this
Hamiltonian to the dynamics of the system remains to be performed. One could proceed with
the quantum mechanical methods used so far, solving for eigenstates of \(\mathbb{\cal
H}_{eff}\). %The approximations taken however allow further simplification. %Säker på att just detta handlar om valet klassisk/kvant?
%Since the slow system is also ''heavy'' it can be considered to be effectively lying in the classical domain,
It is however now practical to consider the slow subsystem to effectively lie in the
classical domain,
and the Hamiltonian derived by quantum mechanical means will be utilized in the role of the
Hamiltonian for classical mechanics. %motivera varför detta är rimligt

Hamilton's canonical equations indicate the time evolution of \(\va{r}\) :%ref Goldstein
\begin{align*} %Fixa massan i nämnarna 
        \frac{d \va{r}}{d t} &= \frac{\partial \mathbb{\cal H}^{(n)}_{eff}}{\partial \va{p}} = \frac{\va{p} -
 \va{A}^{(n)}}{m}\\ %%Inför Hadamard-notation?
                \frac{d \va{p}}{d t}  &= -\frac{\partial \mathbb{\cal H}^{(n)}_{eff}}{\partial \va{r}}=
                \pqty{\frac{\partial
 \va{A}^{(n)}}{\partial \va{r}}}^T \frac{\va{p}-\va{A}^{(n)}}{m} -\frac{\partial \Phi^{(n)}}{\partial \va{r}}
 -\frac{\partial E_n}{\partial \va{r}} = \pqty{\frac{\partial
 \va{A}^{(n)}}{\partial \va{r}}}^T \frac{d \va{r}}{d t} - \frac{\partial
\Phi^{(n)}}{\partial \va{r}} - \frac{\partial E_n}{\partial \va{r}}  
.\end{align*}
Note in particular that the first of these equations imply that the canonical momentum \(\va{p}\) is
\textit{not} \(m \frac{d \va{r}}{d t} \). The effective force acting on the
system can be found, utilizing that the synthetic vector potential does not depend explicitly
on time, i.e., that \(\frac{\partial \va{A}^{(n)}}{\partial t}  = 0\):
\begin{align}\label{eq:dyn1}
        m \frac{d^2 \va{r}}{d t^2} &= \frac{d \va{p}}{d
        t} - \frac{d \va{A}^{(n)}}{d t}  = \pqty{\frac{\partial \va{A}^{(n)}}{\partial \va{r}}}^T \frac{d
\va{r}}{d t} - \pqty{\frac{d \va{r}}{d t}  \vdot \grad}\va{A}^{(n)}
-\frac{\partial \Phi^{(n)}}{\partial \va{r}} - \frac{\partial E_n}{\partial \va{r}}  
.\end{align}
The Jacobian matrix can be treated elementwise, as well as the second term: 
\begin{align*}
        \frac{1}{i\hbar{}}\pqty{\frac{\partial \va{A}^{(n)}}{\partial \va{r}}}_{ji} &= \partial_i
        \bra{n}\ket{\partial_j n} = \bra{\partial_i n}\ket{\partial_j n} +
        \bra{n}\ket{\partial_i \partial_j n}\\
        \frac{1}{i\hbar{}}\pqty{\pqty{\frac{d \va{r}}{d t} \vdot \grad}
\va{A}^{(n)}}_i &= \sum_{j=1}^{5} \frac{d r_j}{d t} \partial_j
                \bra{n}\ket{\partial_i n} =\sum_{j=1}^{5}  \frac{d r_j}{d
                        t}\pqty{ \bra{\partial_j
                n}\ket{\partial_i n} + \bra{n}\ket{\partial_j \partial_i n}}
.\end{align*}
Insertion into equation \ref{eq:dyn1} then yields a higher dimensional analogue to a
cross-product based Lorentz-type force:
\begin{align*}
             \frac{1}{i\hbar{}}\bqty{\pqty{\frac{\partial \va{A}^{(n)}}{\partial \va{r}}}^T \frac{d
\va{r}}{d t} - \pqty{\frac{d \va{r}}{d t}  \vdot \grad}\va{A}^{(n)}}_i
 &= \frac{1}{i\hbar{}}F_{i}^A = \sum_{j=1}^{5} \frac{d
        r_j}{d t} \bqty{\bra{\partial_i n}\ket{\partial_j n}-\bra{\partial_j n}\ket{\partial_i
        n}}\\
        &= \sum_{j=1}^{5} \sum_{l} \frac{d r_j}{d t}\bqty{\bra{\partial_i
                n}\ket{l}\bra{l}\ket{\partial_j n}- \bra{\partial_j
        n}\ket{l}\bra{l}\ket{\partial_i n}}\\
        &= \sum_{j=1}^{5} \sum_{l \ne n} \frac{d r_j}{d t}\bqty{\bra{\partial_i
                n}\ket{l}\bra{l}\ket{\partial_j n}- \bra{\partial_j
        n}\ket{l}\bra{l}\ket{\partial_i n}}\\
        &= \sum_{j\ne i} \sum_{l \ne n} \frac{d r_j}{d t}\bqty{\bra{\partial_i
                n}\ket{l}\bra{l}\ket{\partial_j n}- \bra{\partial_j
        n}\ket{l}\bra{l}\ket{\partial_i n}}
.\end{align*}
Here, \(\ket{l}\) simply denotes an eigenstate to \(\mathbb{\cal H}_{f}\) of some index
\(l\), and the sum over \(l\) is over all available states. The exclusion of \(l=n\)-terms follows as
\(\bra{\partial_i n}\ket{n}\) is purely imaginary, which can be seen from differentiating
\(\bra{n}\ket{n} = 1\). This rearrangement is highly desirable, for it now so happens that this
allows us to take derivatives of the Hamiltonian instead of the rather tricky
differentiation of the eigenkets. Differentiating the Schrödinger equation and acting on it
with some other eigenbra yields:
\begin{align*}
        \mathbb{\cal H}_f \ket{n} &= E_n \ket{n} \implies\\
        \partial \mathbb{\cal H}_f \ket{n} + \mathbb{\cal H}_f \ket{\partial n} &= E_n
        \ket{\partial n} \implies\\
        \bra{l}\partial \mathbb{\cal H}_f \ket{n} &= \bra{l}\ket{\partial n}(E_n - E_l)
.\end{align*}
Rearranging, a very useful relation emerges:
\begin{align}\label{eq:Hdiff}
        \bra{l}\ket{\partial n} &= \frac{\bra{l}\partial \mathbb{\cal H}_f \ket{n}}{E_n
        - E_l}
.\end{align}
This we can insert into the above:
\begin{align}\label{eq:FA}
\frac{1}{i\hbar{}}F_i^A &= \sum_{j\ne i} \sum_{l \ne n} \frac{\frac{d r_j}{d
        t}}{\pqty{E_n-E_l}^2}\bqty{\bra{n}\partial_i \mathbb{\cal H}_f
        \ket{l}\bra{l}\partial_j\mathbb{\cal H}_f\ket{n} - \bra{n}\partial_j \mathbb{\cal
        H}_f\ket{l}\bra{l}\partial_i \mathbb{\cal H}_f\ket{n}}\nonumber\\
        &= 2i\sum_{j\ne i} \sum_{l \ne n} \frac{\frac{d r_j}{d
        t}}{\pqty{E_n-E_l}^2} \Im \bqty{\bra{n}\partial_i \mathbb{\cal H}_f
        \ket{l}\bra{l}\partial_j\mathbb{\cal H}_f\ket{n}}
.\end{align}
Equation \ref{eq:Hdiff} can also be used when evaluating the synthetic
electric potential:
\begin{align}\label{eq:ElPot}
        \Phi^{(n)} &= \sum_{i=1}^{5} \sum_{l \ne n} \frac{\hbar{}^2}{2m_i}\bra{\partial_i
    n}\ket{l}\bra{l}\ket{\partial_i n} = \sum_{i=1}^{5} \sum_{l \ne n}
    \frac{\hbar{}^2}{2m_i}\frac{\bra{n}\partial_i \mathbb{\cal H}_f\ket{l}\bra{l}\partial_i
    \mathbb{\cal H}_f\ket{n}}{\pqty{E_n-E_l}^2}\\
    &= \sum_{i=1}^{5} \sum_{l \ne n}
    \frac{\hbar{}^2}{2m_i}\frac{|\bra{n}\partial_i \mathbb{\cal H}_f\ket{l}|^2}{\pqty{E_n-E_l}^2}
.\end{align}
For the last equalities to hold in both contributions we require that the derivative of the Hamiltonian is
Hermitian, but thankfully derivatives of Hermitian operators are Hermitian so this holds
true.
Note however that no simple form to the derivative of the electric potential \(\Phi^{(n)}\) has
been found, which might not be easily described analytically.

The problem has thus been reduced to evaluating, per equations \ref{eq:FA} and
\ref{eq:ElPot},
\begin{align}\label{eq:dynfin}
        m\frac{d^2 \va{r}}{d t^2} &= \va{F}^A-\frac{\partial \Phi^{(n)}}{\partial
        \va{r}} - \frac{\partial E_n}{\partial \va{r}} 
.\end{align}
\subsection{Differentiation of the Hamiltonian}\label{sec:Hdiff}
In order to easily evaluate equations \ref{eq:FA} and \ref{eq:ElPot} derivatives of
\(\mathbb{\cal H}_f\) from equation \ref{eq:Hf} are to be found. Writing any of the coordinates \(x\), \(y\), \(z\)
as \(r\) the derivatives can
be written:
\begin{align}\label{eq:diffham}
        \partial_r\mathbb{\cal H}_f  &= \gamma B \hbar{} \pmqty{
 \frac{\dot B}{B}\cos(\vartheta_B) -\dot \vartheta_B \sin(\vartheta_B) & \Omega & 0 \\
 \Omega^{*} & 0 & \Omega\\
 0 & \Omega^{*}& -\frac{\dot B}{B}\cos(\vartheta_B) + \dot \vartheta_B \sin(\vartheta_B)
        }, \nonumber\\
        \partial_{\vartheta_r} \mathbb{\cal H}_f &= \gamma B \xi \hbar{}\pmqty{
                -\sin(2\vartheta_r) & -\sqrt{2}  e^{-i\varphi_r}\cos(2\vartheta_r) &
                 e^{-2i\varphi_r}\sin(2\vartheta_r)\\
                -\sqrt{2}  e^{i\varphi_r}\cos(2\vartheta_r) & 2\sin(2\vartheta_r) &
                \sqrt{2} e^{-i\varphi_r}\cos(2\vartheta_r)\\
                 e^{2i\varphi_r}\sin(2\vartheta_r) & \sqrt{2} 
                e^{i\varphi_r}\cos(2\vartheta_r) & -\sin(2\vartheta_r)
        }, \nonumber\\
        \partial_{\varphi_r} \mathbb{\cal H}_f &= \gamma B \xi \hbar{}\pmqty{
                0 & i  \frac{e^{-i\varphi_r}}{\sqrt{2} }\sin(2\vartheta_r) &
                -2i e^{-2i\varphi_r}\sin[2](\vartheta_r)\\
                -i  \frac{e^{i\varphi_r}}{\sqrt{2} }\sin(2\vartheta_r) & 0 & -i 
                \frac{e^{-i\varphi_r}}{\sqrt{2} }\sin(2\vartheta_r) \\
                2i e^{2i\varphi_r}\sin[2](\vartheta_r) & i \frac{e^{i\varphi_r}}{\sqrt{2}
                }\sin(2\vartheta_r) & 0
        }
.\end{align}
Here, \(\Omega = (-\frac{\dot B}{B}\sin(\vartheta_B) + i\dot \varphi_B 
 \sin(\vartheta_B) -\dot \vartheta_B \cos(\vartheta_B)) \frac{e^{-i\varphi_B}}{\sqrt{2}}\) is introduced as a means of compressing the
rather lengthy expressions for the derivative with respect to \(r\).
\subsection{Solution of the fast subsystem} 
The usage of the Born-Oppenheimer approximation requires a solution for the fast subsystem,
i.e., that the eigenvalues and eigenvectors of \(\mathbb{\cal H}_f\) are found.
Unfortunately this is not possible analytically for the present system, but note that it
is the same as solving the following cubic characteristic equation, which follows
from equation \ref{eq:Hf}, for the
eigenvalues \(\lambda_n = \frac{E_n}{\gamma B\hbar{}}\): %Citera Sjöqvist, Yi
\begin{align}\label{eq:lambda}
        0 &= \mdet{
                 \xi\cos[2](\vartheta_r) + \cos(\vartheta_B) - \lambda_n& -\frac{
                        e^{-i\varphi_B}}{\sqrt{2}
                }\sin(\vartheta_B) - \xi\frac{e^{-i\varphi_r}}{\sqrt{2} }\sin(2\vartheta_r) &
                \xi e^{-2i\varphi_r}\sin[2](\vartheta_r)\\
                 -\frac{e^{i\varphi_B}}{\sqrt{2}}\sin(\vartheta_B) -
                \xi \frac{e^{i\varphi_r}}{\sqrt{2} }\sin(2\vartheta_r) &
                -\xi\cos(2\vartheta_r) - \lambda_n & -\frac{
                e^{-i\varphi_B}}{\sqrt{2} }\sin(\vartheta_B)+ \xi\frac{e^{-i\varphi_r}}{\sqrt{2}
        }\sin(2\vartheta_r)\\
                \xi e^{2i\varphi_r}\sin[2](\vartheta_r) & -\frac{
                e^{i\varphi_B}}{\sqrt{2} }\sin(\vartheta_B) +
                        \xi \frac{e^{i\varphi_r}}{\sqrt{2} }\sin(2\vartheta_r) &
                        \xi \cos[2](\vartheta_r) - \cos(\vartheta_B) - \lambda_n 
        }
.\end{align}
Calculation of eigenvalues and eigenvectors may be left to numerics from
the onset. Equation \ref{eq:lambda} could in principle be differentiated implicitly to
receive the derivatives of the energies also needed, but since simulation of the system requires many other quantities 
to be calculated numerically, the derivatives of the energies will be done likewise for
the sake of practicality.
%Derivatives of \(E_n\) are however needed, and the possibility of calculating
%these analytically for each point in parameter space without requiring the calculation of
%neighbouring eigenvalues is an alluring one. Numerical approximation of the derivatives
%would require the simultaneous calculation of neighbouring eigenvalues, and the imposing
%task of deriving equation \ref{eq:lambda} is motivated.


\end{document}
