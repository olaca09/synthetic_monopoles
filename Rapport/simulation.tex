\documentclass[main.tex]{subfiles}

\begin{document}
\section{Simulation}\label{sec:simulation}
The equation of motion \ref{eq:dynfin} represents the furthest point that analytical
methods have
reached for this system. The inherent complexity in the problem now necessitates numerical
analysis if further results are to be reached, and for this reason the resultant dynamics
of the translating and rotating dumbbell have been simulated for select parameters in a predetermined external magnetic field.

\subsection{The chosen field}\label{sec:chofield}
It is apparent from the form of the differentiated Hamiltonians of equations
\ref{eq:diffham1}, \ref{eq:diffham2} and \ref{eq:diffham3} that the synthetic fields are
set in proportion to the inhomogeneity of the external magnetic field. To maximize the
action of the synthetic fields it then becomes important to consider an external field
which varies as much as possible in both magnitude and direction. It is furthermore of
interest that the field contains a point or points of zero magnitude. If the spin-spin
interaction factor \(J\) is zero, these will be the points of synthetic magnetic charge
which we wish to study. In general we consider \(J \ne 0\), for which these charges will
not sit precisely at the points of vanishing fields but will translate as a function of
\(J\) away from these as described in section \ref{sec:regmono}.

As an initial example of such a field consider two simple coils of the same diameter and
axis of symmetry placed some distance apart. If a current is run through both coils but
the direction of current is different between the two, one running clockwise and the other
counter-clockwise, a suitable field is created. Not even the complete field of a single
coil has a closed analytic expression, but we may still note some qualitative properties.
If the axis of rotational symmetry for both coils is taken to be the \(z\)-axis, as will be
done in the simulations, the \(xy\)-plane at equal distance to both coils will have a field
\(z\)-component that is zero. This follows from simple symmetry reasons, from which it is
also apparent that the total field is precisely zero at the ''centre'' of this plane, where
the distance to all current-carrying wires are the same. The magnetic field in the plane then increases
in magnitude as the distance between \(xy\)-plane and the nearest coil wire decreases,
pointing either directly towards or away from the centre point depending on current
direction. In the limit of increasing distance to the centre point the field naturally
tends towards zero. If points away from the \(xy\)-plane are considered the field will tend
in a smooth fashion towards the regular single coil field.

For computational practicality the coils for the simulation are taken to be as described
above, with the axis of symmetry being the \(z\)-axis, but being of square form instead of
the regular circular shape. This perturbs the field described above slightly, which after
all does nothing but increase the desired inhomogeneity. The side length of the coils is
taken to be the distance between the coils, so that they form edges of a cube.

\subsection{Code outline}\label{sec:code}
Simulation of the system is in essence nothing more complicated that solving the
ordinary differential equation, the ODE, \ref{eq:dynfin}. Some difficulties arise since the fast
Hamiltonian is not analytically diagonalisable, all scalar fields must be calculated and in
addition several quantities need to be numerically differentiated. The dumbbell is given an
initial velocity and is positioned close to the boundary of a box of side length \(\frac{2}{3}\) times the
distance between the coils, centred about the zero-field point described in section
\ref{sec:chofield}. All simulation is done in the box. That is to say, that the dumbbell is
only allowed inside of this box and that field values are exclusively calculated within the
box, which will be at times referred to as the ''lab''.

For complete Python scripts, see appendix \ref{app:scripts}. A few notable choices of
methods will be here mentioned, but further details should be discernible from the
commented code.

The lab is divided into a discrete lattice of points so that field values can be
calculated in advance and saved to file. This has the side-effect of restricting the
position of the dumbbell to points in this lattice. The field generation is done per numerical
integration of the Biot-Savart law using the same step size as the lab lattice.

Solution of the differential equation is done by means of the \textit{scipy} function
solve\_ivp() which is a somewhat sophisticated tool containing error estimation and
flexible termination conditions, here used to stop the algorithm if the dumbbell leaves the
lab. As is often done the ODE is transformed into first order by extension of the
five positional coordinates to a ten-dimensional position and velocity vector. All
acceleration contributions are summed up in a single function called by the ODE solver. 
As part of this process the diagonalization of the fast Hamiltonian is performed by the
\textit{scipy} funtion eigh(). It here becomes crucial to consider which eigenstate of the
fast Hamiltonian the dumbbell is in. A predetermined eigenstate is selected for each
simulation, indexed by \(n = 0\), \(1\), \(2\) in increasing order of energies. Since each
eigenstate is indexed by the value of its energy it is crucial that the energies never
cross, which would lead to eigenstates changing index. This problem is of course resolved
by the adiabatic approximation assumed, and any trajectories of dumbbells traversing points
of energy degeneracy are discarded.

At many stages in the process are derivatives of fields needed, as well as derivatives of
the fast energies. Since no closed forms of these quantities are available the derivatives
of some quantity \(\Lambda (\omega)\) depending on coordinate \(\omega\) has been
approximated as \[
\frac{\partial \Lambda}{\partial \omega} \approx \frac{\Lambda(\omega + s_\omega) -
\Lambda(\omega - s_\omega}{2s_\omega}
,\] 
where \(s_\omega\) is the step size of \(\omega\) implied by the lattice.
This is sometimes called the three-point centred difference formula. No error estimation scheme
has been implemented for these parts of the algorithm, so care must be taken that the step size does not
become so small that rounding errors of the floating point operators dominate.
A rule of thumb is that a step size of \(s_\omega \approx \sqrt[3]{\epsilon}\omega_c\) is
close to the optimal point of good precision without leading to large rounding
errors \cite{numerical}. Here \(\epsilon\) is the machine
epsilon of order \(10^{-16}\), and \(\omega_c\) is the typical scale of \(\omega\), which
is taken to be the lab side length, so it is clear that any lattice with less than \(10^5\) sites
along each cube side leads to step sizes well above this limit.

Finally the result is plotted using appropriate functions, as of yet a simple
\textit{pyplot} implementation has been done.
\end{document}
