\documentclass[main.tex]{subfiles}
\begin{document}
\section{Conclusions}
The plausibility of synthetic field effects from the Born-Oppenheimer approximation
playing a part in the dynamical evolution of the dumbbell model has been explored. Complete
equations of motion, including the rotational degrees of freedom, have been found. These
contain a Lorentz-type force involving a five-dimensional generalization of the cross
product, which depends on the synthetic magnetic field emanating from the monopoles that
are to be studied. Numerical evaluation of these equations of motion has been made
possible through the development of Python scripts, one of the main achievements of the
project.

Sample simulations by these scripts for a small set of parameters have been performed. The
main results of these include a demonstration of the repulsive nature of the synthetic
scalar field of the Born-Oppenheimer approximation, albeit for unrealistically small
masses. For a much more reasonable mass range, in the order of the atomic mass unit, the
simulations have demonstrated noticeable impact of both synthetic scalar and magnetic
fields on the resultant dynamics for the high energy eigenstate of the spin system. The nature of these contributions has not been explored
in any qualitative nor quantitative manner, but investigations into such matters have been facilitated by the construction of
the scripts.

The promise of observing precisely the monopole effects of the synthetic magnetic field
has been discussed with its limitations and difficulties. In addition to theoretical
complications the endeavour is currently limited by the numerical performance of the
scripts developed. Thus, optimization and further developments of these are warranted, as
is further research into proposing experimental realisations of the system herein modelled. 
\end{document}
