\documentclass[main.tex]{subfiles}

\begin{document}
\vspace*{\fill}
\begin{abstract}
        In the Born-Oppenheimer approximation for a quantum system the emergent synthetic
        magnetic field can be seen as generated by monopoles at points of degeneracy, in
        full analogue to the synthetic fields generating the geometric phase of
        adiabatically evolving quantum systems. The plausibility of using these synthetic
        magnetic monopoles as a means to study magnetic monopole dynamics in the absence of
        fundamental magnetic monopoles has been explored. A bipartite spin system
        consisting of a dumbbell translating and rotating through space has been modelled,
        and full equations of motion in the presence of an external magnetic field have
        been derived. A collection of scripts for numerical evaluation of these equations
        of motion were subsequently developed, and further put to use in sample simulations
        for a small range of parameters. The results demonstrate non-negligible
        perturbations to the centre of mass motion when compared to motion not considering
        the Born-Oppenheimer synthetic fields, for dumbbell masses of small but not
        unrealistic proportions. The problems inherent in this approach to elucidating
        motion in magnetic monopole fields are discussed, but the method should not yet
        be dismissed until further investigations have been made.
\end{abstract}

\begin{otherlanguage}{swedish}
\begin{abstract}
        Under Born-Oppenheimer-approximationen för ett kvantsystem kan det emergenta
        syntetiska magnetfältet ses som alstrat av monopoler vid degenerationspunkter, helt
        analogt med de syntetiska fält som genererar den geometriska fasen vid adiabatisk
        utveckling av kvantsystem. Möjligheten att använda dessa syntetiska magnetiska
        monopoler för att studera dynamiken från verkan av en magnetisk monopol, trots att
        fundamentala magnetiska monopoler ej observerats, har utforskats. Ett tvådelat
        spinsystem beståendes av en hantel som translaterar och roterar genom rummet har
        modellerats, och fullständiga rörelseekvationer i närvaron av ett yttre magnetfält
        har härletts. Kod till ändamålet att utvärdera dessa rörelseekvationer har därpå
        utvecklats, och vidare nyttjats för att simulera rörelsen för ett stickprov av
        parametrar. Resultaten visar på ej försumbara perturbationer av masscentrums
        rörelse vid jämförelse med rörelse utan hänsyn till de syntetiska
        Born-Oppenheimer-fälten, för hantlar av liten men inte orealistiskt liten massa.
        Problemen och komplikationerna för det här angreppssättet till att utforska rörelse genom
        magnetiska monopolers fält diskuteras, men metoden bör ej ännu avvisas innan vidare
        undersökning har genomförts.
\end{abstract}
\end{otherlanguage}
\vspace*{\fill}
\end{document}
